\documentclass[a4paper, 12pt]{article}
% \usepackage[brazil]{babel}
% \usepackage[utf8]{inputenc}

\usepackage{sbc-template}

\usepackage{graphicx,url}

\usepackage[utf8]{inputenc}
\usepackage[portuguese]{babel}
\usepackage[utf8]{inputenc}
\usepackage[T1]{fontenc}
\usepackage{listings}
\usepackage{xcolor}

\usepackage{inconsolata}
\lstset{
    language=C, %% Troque para PHP, C, Java, etc... bash é o padrão
    basicstyle=\ttfamily\small,
    numberstyle=\footnotesize,
    numbers=left,
    backgroundcolor=\color{gray!10},
    frame=single,
    tabsize=2,
    rulecolor=\color{black!30},
    title=\lstname,
    escapeinside={\%*}{*)},
    breaklines=true,
    breakatwhitespace=true,
    framextopmargin=2pt,
    framexbottommargin=2pt,
    inputencoding=utf8,
    extendedchars=true,
    literate={á}{{\'a}}1 {ã}{{\~a}}1 {é}{{\'e}}1,
}


\sloppy

\title{Multiplicação de um vetor por $\pi$ sequencialmente e paralelamente}

\author{Diego Luiz N. Gonçalves\inst{1}, Rafael H. Bordini\inst{2}, Flávio Rech
  Wagner\inst{1}, Jomi F. Hübner\inst{3} }


\address{Universidade Federal de Mato Grosso do Sul- Campus de Ponta Porã
  (UFMS-CPPP)\\
  Caixa Postal 15.064 -- 91.501-970 -- Porto Alegre -- RS -- Brazil
\email{\ diegoreke@hotmail.com, jomi@inf.furb.br,funalo@gmail.com}
}

\begin{document} 

\maketitle
 
\begin{resumo} 
  O trabalho realizado tem como objetivo analisar a diferença do tempo de execução entre um código sequêncial e um código paralelo, utilizando a API de desenvolvimento CUDA, e a linguagem de programação C, na resolução do problema onde deve-se multiplicar todos os elementos de um vetor de tamanho $2^n$, onde $n$ começa em $10$ e termina em $19$; pelo número $\pi$ com precisão de $4$ casas decimais.
\end{resumo}

\section{Codigo Sequencial}
\begin{lstlisting}
/*PROGRAMA SEQUÊNCIAL*/

#include <stdio.h>
#include <time.h> //biblioteca para calcular o tempo
#include <stdlib.h>

#define max 32768 //tamanho do vetor 
#define pi 3.1415 // numero pi com 4 casas decimais

int main(){
    /*Variável que contabilizará o tempo gasto*/
    double time=0.0; 
    /*Vetor onde serão realizado os cálculos*/
    double vet[max]; 
    
    /*Variável do tipo clock_t, que inicia a contagem do tempo*/
    clock_t begin=clock(); 
    
    /*Início do processamento*/
    for(long int i=0;i<max;i++){ 
        vet[i]= i*pi;
    }
    /*Fim do processamento*/
    
    /*Variável do tipo clock_t, que termina a contagem do tempo*/
    clock_t end=clock(); 
    
    /*Calculando o tempo em segundos*/
    time+= (double)(end - begin) / CLOCKS_PER_SEC;
    
    /*Mostrando o tempo final*/
    printf("Tempo gasto: %f segundos", time); 
    return 0;
}
\end{lstlisting}

\section{Código em paralelo}
\begin{table}[!h]
\centering
\begin{tabular}{|l|l|l|l|}
\hline
       Entrada & Tempo Sequêncial & Tempo Paralelo & Speed up \\ \hline
                     $2^{10}$ & 0,0000066$s$ &  &  \\ \hline
                     $2^{11}$ & 0,0000084$s$ &  &  \\ \hline
                     $2^{12}$ & 0,0000254$s$ &  &  \\ \hline
                     $2^{13}$ & 0,0000485$s$ &  &  \\ \hline
                     $2^{14}$ & 0,0001018$s$ &  &  \\ \hline
                     $2^{15}$ & 0,0002044$s$ &  &  \\ \hline
                     $2^{16}$ & 0,0003999$s$ &  &  \\ \hline
                     $2^{17}$ & 0,0008107$s$ &  &  \\ \hline
                     $2^{18}$ & 0,0016843$s$ &  &  \\ \hline
                     $2^{19}$ & 0,0033249$s$ &  &  \\ \hline
                    
\end{tabular}
\caption{Tabela comparativa entre o tempo sequêncial vs. tempo paralelo e o speed up}
\end{table}

\section{References}

Bibliographic references must be unambiguous and uniform.  We recommend giving
the author names references in brackets, e.g. \cite{knuth:84},
\cite{boulic:91}, and \cite{smith:99}.

The references must be listed using 12 point font size, with 6 points of space
before each reference. The first line of each reference should not be
indented, while the subsequent should be indented by 0.5 cm.

\bibliographystyle{sbc}
\bibliography{sbc-template}

\end{document}

