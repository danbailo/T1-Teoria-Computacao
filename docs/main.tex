\documentclass[a4paper, 12pt]{article}
\usepackage[brazil]{babel}
\usepackage[utf8]{inputenc}
\usepackage{indentfirst}
\usepackage{graphicx}
\usepackage{graphics}

\usepackage{tabularx}
\usepackage{graphicx}
\usepackage{adjustbox}

\usepackage{booktabs}
\usepackage[font=footnotesize,labelfont=bf]{caption} %muda o tamanho das caption e deixa em negrito
\usepackage{cite}
\usepackage{color}   %May be necessary if you want to color links
\usepackage{hyperref}


\hypersetup{
    colorlinks=true,
    citecolor=blue,
    filecolor=blue,
    linkcolor=blue,
    urlcolor=blue
}
\usepackage{listings}
\renewcommand{\lstlistingname}{Algoritmo}
% \definecolor{codegreen}{rgb}{0,0.6,0}
\definecolor{codegray}{rgb}{0.5,0.5,0.5}
% \definecolor{codepurple}{rgb}{0.58,0,0.82}
\definecolor{backcolour}{rgb}{0.95,0.95,0.92}
\lstdefinestyle{mystyle}{
    backgroundcolor=\color{backcolour},   
    % commentstyle=\color{codegreen},
    % keywordstyle=\color{magenta},
    numberstyle=\tiny\color{codegray},
    % stringstyle=\color{codepurple},
    basicstyle=\footnotesize,
    breakatwhitespace=false,         
    breaklines=true,                 
    captionpos=b,                    
    keepspaces=true,                 
    numbers=left,                    
    numbersep=5pt,                  
    showspaces=false,                
    showstringspaces=false,
    showtabs=false,                  
    tabsize=2
}
\lstset{style=mystyle}

\begin{document}
\begin{titlepage}
    \begin{center}
		\LARGE{Universidade Federal de Mato Grosso do Sul}\\
		\vspace{5pt}
        \large{Campus Ponta Porã}\\ 
        \large{{\textbf{Teoria da Computação}}}\\ 
        \vspace{15pt}
        \vspace{95pt}
        \textbf{\large{Trabalho Prático I}}\\
        \vspace{15pt}
        \textbf{\LARGE{Uma Heurística Gulosa e uma Heurística GRASP para o
        Problema da Mochila}}\\
        %\title{{\large{Título}}}
        \vspace{3,5cm}
    \end{center}
    
    \begin{flushleft}
        \begin{tabbing}
            Aluno: Daniel de Leon Bailo da Silva\\            
            Professor: Eduardo Theodoro Bogue\\
            %Professor co-orientador: \\
    \end{tabbing}
 \end{flushleft}
    \vspace{1cm}
    
    \begin{center}
        \vspace{\fill}
            Agosto\\
         2019
            \end{center}
\end{titlepage}

\clearpage
\tableofcontents
\thispagestyle{empty}
\clearpage

% ref
% https://en.wikibooks.org/wiki/LaTeX/Labels_and_Cross-referencing
% tabela
% https://www.tablesgenerator.com/#

\pagenumbering{arabic}
\section*{Resumo}
\addcontentsline{toc}{section}{Resumo}
\label{sec:resumo}
Este trabalho consiste em mostrar os resultados obtidos a partir da execução
do algoritmo da {\it Mochilha Boolena} ou {\it Knapsack 0/1}, em suas versões dinâmicas.

Feito isso, dada as instâncias contento os dados necessários para aplicar os algoritmos, foi comparado o tempo de execução para cada instância 
nas suas versões dinâmicas, {\it Top-Down} e  {\it Bottom-Up}.\\

{\bf Considerar o seguinte ambiente para a obtenção dos resultados:}
\begin{itemize}
    \item Processador: Intel Core™ i5-8250U
    \begin{itemize}
        \item Número de núcleos 4
        \item Número de threads 8
        \item Frequência baseada em processador 1.60 GHz
        \item Frequência turbo max 3.40 GHz
    \end{itemize}
    \item Memória: 8GB RAM
\end{itemize}
Este trabalho foi armazenado num repositório {\it GitHub} para melhor controle do versionamento do programa.\\
\url{https://github.com/danbailo/T2-Analise-Algoritmos-I}
\clearpage

\section{Introdução}
\clearpage

\section{PROBLEMA}

\begin{table}[h]
    % \centering
    \begin{tabular}{|r|l|}
        \hline
        \textbf{Instâncias} &  \textbf{Valores} \\ \hline
        input1.in  &    12840 \\ \hline
        input2.in  &    19687 \\ \hline
        input3.in  &    39665 \\ \hline 
        input4.in  &    39578 \\ \hline
        input5.in  &    21019 \\ \hline
        input6.in  &    64727 \\ \hline
        input7.in  &     2129 \\ \hline
        input8.in  &     1017 \\ \hline
        input9.in  &    19976 \\ \hline
        input10.in &    39897 \\ \hline
        input11.in &    59836 \\ \hline
        input12.in &    49988 \\ \hline
        input13.in &    59990 \\ \hline
        input14.in &    20820 \\ \hline
        input15.in &    20676 \\ \hline
        input16.in &     3995 \\ \hline
    \end{tabular}
\end{table}

\clearpage

\section{Análise dos Resultados}


\begin{table}[h]
\centering
\caption{My caption}
\label{my-label}  
\begin{adjustbox}{width=1\textwidth}  
    \centering
    \begin{tabular}{|l|l|l|}
    \hline
    Item & Peso em kg (Tamanho) & Valor\\ \hline
    1 & 50 & 100\\ \hline
    2 & 20 & 60\\ \hline
    3 & 10 & 40\\ \hline
    4 & 40 & 40\\ \hline
    5 & 8 & 4 \\ \hline
    \end{tabular}
\end{adjustbox}
\end{table}

\clearpage

\end{document}